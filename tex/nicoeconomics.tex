\newcommand{\BRL}{\text{R}\$}

%\game: fazer matrix de payoff bonita
\newcounter{nblin}
\newcounter{nbcol}
\newcounter{nbcoltwo}
\newcounter{contest}
\newcounter{contpay}

% \game[Alice, Bob][{Up, Middle, Down},{Left, Center, Right}][{3/3,1/2,0/2},{2/1,2/2,-1/3},{2/0,3/-1,-1/-1}]
\newcommand{\gameCell}[2]{({\color{blue} #1},{\color{red} #2})}
\newcommand{\gamePlayer}[2]{{\color{blue} #1}{\color{red} #2}}
\makeatletter
\def\game[#1,#2][#3,#4][#5]{
    \setcounter{nblin}{0}
    \setcounter{nbcol}{0}
    \setcounter{nbcoltwo}{0}
    \setcounter{contest}{0}
    \setcounter{contpay}{0}

    % #1 = Jogador 1
    % #2 = Jogador 2
    % #3 = Estratégias de 1
    % #4 = Estratégias de 2
    % #5 = Matrix de payoffs
    % #6 = bool(0: se sem linhas, 1: com linhas e na diagonal)
    \foreach \j in {#3} {

      \addtocounter{nblin}{1}
    }
    \foreach \j in {#4} {
        \addtocounter{nbcol}{1}
    }
    \addtocounter{nbcol}{2}
    \setcounter{nbcoltwo}{\thenbcol}
    \addtocounter{nbcol}{-2}

    \def\tablehead{%
    \multicolumn{2}{c}{} & \multicolumn{\thenbcol}{c}{{\color{red}\textbf{#2}}}\\\cline{3-\thenbcoltwo} \multicolumn{1}{c}{}&}
    %
        \foreach \lhs in {#4}{
              \protected@xappto\tablehead{ & {\color{red}\textit{\lhs}}}
    }

      %\addtocounter{nbcol}{2}
    %\protected@xappto\tableheader{ \\\cline{3-\thenbcol}}
    %\addtocounter{nbcol}{-2}
    \def\tabledata{\multirow{\thenblin}*{{\color{blue}\textbf{#1}}}}% reset \tabledata

        \setcounter{contest}{0}
        \foreach \j in {#3}{
          \addtocounter{contest}{1}
              \protected@xappto\tabledata{& {\color{blue}\textit{\j}}}
              \setcounter{contpay}{0}
              \foreach \l in {#5}{
                \addtocounter{contpay}{1}
                \ifnum \thecontest=\thecontpay{     
                  \foreach \lhs/\rhs in \l {% build table data from #1
                    \protected@xappto\tabledata{ & {\color{blue}\lhs},\;{\color{red}\rhs}}
                    }
                  \gappto\tabledata{\\}
                }
                \else{}\fi
             }
        }

        \begin{table}[H]
          \centering
          \begin{tabular}{c|c|*{\thenbcol}{c}|}
              \tablehead\\\cline{2-\thenbcoltwo}
              \tabledata\cline{2-\thenbcoltwo}
          \end{tabular}

        \end{table}
}
\makeatother


\newenvironment{economics_diagram}[4]{
\begin{figure}[H]
    \centering
    \begin{tikzpicture}
        \begin{axis}[
              scale only axis, 
              grid=major, 
              grid style={dashed, gray!20},
              %minor x tick num=5,
              %minor y tick num=5,
              axis x line=center,
              axis y line=center, 
              xlabel style={below right},
              ylabel style={above left},
              xmin=0,
              ymin=0,
              xlabel={#1}, %Variável
              ylabel={#2}, %Variável           
              xmax={#3*1.05},  
              ymax={#4*1.05}
              ]
              \path [name path=xaxis]
                      (\pgfkeysvalueof{/pgfplots/xmin},0) --
                      (\pgfkeysvalueof{/pgfplots/xmax},0);
                      
              \pgfplotsset{domain = 0:{#3*1.05}};

        }{
        \end{axis}
    \end{tikzpicture}
\end{figure}
}

\newenvironment{edgeworth_box}[4]{
\begin{figure}[H]
    \centering
    \begin{tikzpicture}
        \begin{axis}[
              scale only axis, 
              grid=major, 
              grid style={dashed, gray,opacity=.2},
              %minor x tick num=5,
              %minor y tick num=5,
              xmin=0,
              ymin=0,
              xlabel={#1}, %Variável
              ylabel={#2}, %Variável           
              xmax={#3},  
              ymax={#4},
              view={0}{90}
              ] 
              \pgfplotsset{domain = 0:{max(#3,#4)*1.05}};
            \addplot[color=black,mark=*,mark size = 1] (0,0) node[above right] {Alice};           
            \addplot[color=black,mark=*,mark size = 1] (#3,#4) node[below left] {Bob};
         
        }{
        \end{axis}
    \end{tikzpicture}
\end{figure}
}


\newcommand{\addLabel}[3]{
    \node at (#1,#2) {\footnotesize #3};
 }

\NewDocumentCommand{\addIC}{O{red}O{red}mm}{
    \addplot3 [
    name path global=#2,
    thick,
    contour gnuplot={
        levels=0,
        labels=false,
        draw color=#1
    },
    samples=25
    ] {#3-(#4)};
}

\NewDocumentCommand{\addPlot}{O{red}O{red}m}{
    \addplot[name path=#1,samples=50,#2,thick,smooth]{#3};
}

\NewDocumentCommand{\addArea}{O{red}mmm}{
    \addplot[#1, opacity=0.2] fill between [of=#2 and #3, soft clip={domain=#4}];
}

\NewDocumentCommand{\addDot}{O{}O{below left}smm}{
    \addplot[color=black,mark=*,mark size = 1] (#4,#5) node [#2] {\footnotesize #1};
    \IfBooleanTF{#3}{}{\draw[dotted,thick] (0,#5)--(#4,#5) --(#4,0);}
}

\NewDocumentCommand{\addVector}{O{0,0}O{}m}{
    \draw[thick,->] (#1)--(#3) node[anchor=north west] {#2};
}

\newenvironment{landscape}{
    \newpage
    \KOMAoptions{paper=landscape,pagesize}
    }{
    \recalctypearea
    \KOMAoptions{paper=portrait,pagesize}
    \recalctypearea
    }